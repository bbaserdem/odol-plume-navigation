\title{SPH Notes}
\author{
    Batuhan Başerdem  \\
    Department of Physics\\
    Stony Brook University\\
}
\date{\today}

\documentclass[12pt]{article}

\usepackage[margin=0.5in]{geometry}
\usepackage{nicefrac}

\begin{document}
\maketitle

\begin{abstract}
These are my notes for Fluid Mechanics Simulation using SPH
\end{abstract}


\section{Quantities}

Mass is measured per transverse lenght in 2D.

\section{Kernel}

The following is how the kernel function is abstracted

\[
    w^{(k)}_h(\tilde{\mathbf{r}}) = \frac{f^{k}(q)}{h^{d+k}} 
\]

\[
    q = \frac{\left|\tilde{\mathbf{r}}\right|}{h}
\]

\[
    \nabla w^{(k)}_h(\tilde{\mathbf{r}}) = w^{(k)}_h(\tilde{\mathbf{r}})
        \frac{\tilde{\mathbf{r}}}{\left|\tilde{\mathbf{r}}\right|}
\]

\[
    \tilde{\mathbf{r}}_{ab} = \mathbf{r_a} - \mathbf{r_b}
\]

The function f here is up to choice, and determines the coefficients.
I will always default to Wendland kernel,
as it is the most regular one from the literature I have been reading.

\[
    f_{W,d}(q) = a_{W,d} \left\{
        \begin{array}{cl}
            {\left(1-\frac{q}{2}\right)}^{4} \left( 1 + 2q \right) &
            \mbox{ 0 \geq q \geq 2 } \\
            0 &
            \mbox{ 2 < q }
        \end{array} \right.
\]

\[
    f^{'}_{W,d}(q) = a_{W,d} \left\{
        \begin{array}{cl}
            {\left(1-\frac{q}{2}\right)}^{3} \left( -5q \right) &
            0 \geq q \geq 2 \\
            0 &
            2 < q
        \end{array} \right.
\]

\[
    \alpha_{W,2} = \frac{7}{4\pi}
\]

\[
    \alpha_{W,2} = \frac{21}{16\pi}
\]

\section{Solid Wall Particles}

Walls can be made with solid particles.
Paper I'm following uses twice as dense particles for the edges.
The type of force used is the Lennard-Jones potential;

\[
    \mathbf{F}_{a,b} = \left\{
        \begin{array}{cl}
            D \left[ { \left( \frac{r_0}{r_{ab}} \right) }^{n_1} -
                     { \left( \frac{r_0}{r_{ab}} \right) }^{n_2} \right] &
            r_{ab} < r_0 \\
            0 &
            r_0 \leq r_{ab} 
        \end{array} \right.
\]

Which has 4 parameters. ($n_1=12$, $n_2=4$, $D$, $r_0$)

\section{Integration Scheme}

In terms of numerical simulations, a symplectic scheme seems to be the best.
I will usually default to the Störmer-Verlet scheme

\[
    \mathbf{r}_{a}^{m+\nicefrac{1}{2}} =
    \mathbf{r}_{a}^{m} +
    \mathbf{u}_{a}^{m} \frac{\delta t}{2}
\]
\[
    \mathbf{u}_{a}^{m+1} =
    \mathbf{r}_{a}^{m} +
    \frac{1}{m_a} \mathbf{F}_{a}^{m+\nicefrac{1}{2}} \delta t
\]
\[
    \mathbf{r}_{a}^{m+1} =
    \mathbf{r}_{a}^{m+\nicefrac{1}{2}} +
    \mathbf{u}_{a}^{m+1} \frac{\delta t}{2}
\]

\section{Interpolators}

There are multiple interpolation operators, and different ones can be used.
All these formulas refer to the volume; which is

\[
    V_a = \frac{m_a}{\rho_a}
\]

Field interpolator is simple;

\[
    J \left\{ A_a \right\} := \left\{
        \sum\limits_{b} V_b A_b w_{ab}
    \right\} \approx
    \left\{ A_a \right\}
\]

There are specific operators, gradient and divergence for tensors.

\[
    \mathbf{G} \left\{ A_a \right\} := \left\{
        \sum\limits_{b} V_b A_b w^{'}_{ab} \mathbf{e}_{ab}
    \right\} \approx
    \left\{ \nabla A_a \right\}
\]

\[
    D \left\{ \mathbf{A}_a \right\} := \left\{
        \sum\limits_{b} V_b \mathbf{A}_b \cdot w^{'}_{ab} \mathbf{e}_{ab}
    \right\} \approx
    \left\{ \nabla \cdot \mathbf{A}_a \right\}
\]

\subsection{Variants of Interpolation Operators}

There are operations that leave the differential operators invariant,
if applied on the continuous fields.
New types of gradient fields can be constructed out of these.

\[
    \mathbf{G}^k_a \left\{ A_i \right\} := \left\{
        \sum\limits_{b} V_b
        \frac{\rho_b^{2k}A_a+\rho_a^{2k}A_b}{{\left(\rho_a\rho_b\right)}^k}
        w^{'}_{ab} \mathbf{e}_{ab}
    \right\} \approx
    \left\{ \nabla A_a \right\}
\]

\[
    \widetilde{\mathbf{G}}^k_a \left\{A_i\right\} := -\frac{1}{\rho_a^{2k}}\left\{
        \sum\limits_{b} V_b {\left( \rho_a \rho_b \right)}^{k}
        \left( A_a - A_b \right) w^{'}_{ab} \mathbf{e}_{ab}
    \right\} \approx
    \left\{ \nabla A_a \right\}
\]

And the new divergence operators are;

\[
    D^k_i \left\{ \mathbf{A}_a \right\} := \left\{
        \sum\limits_{b} V_b
        \frac{\rho_b^{2k}\mathbf{A}_a+\rho_a^{2k}\mathbf{A}_b}{{\left(\rho_a\rho_b\right)}^k}
        \cdot w^{'}_{ab} \mathbf{e}_{ab}
    \right\} \approx
    \left\{ \nabla \cdot \mathbf{A}_a \right\}
\]

\[
    \widetilde{D}^k_i \left\{ \mathbf{A}_a \right\} := -\frac{1}{\rho_a^{2k}}\left\{
        \sum\limits_{b} V_b {\left( \rho_a \rho_b \right)}^{k}
        \left(\mathbf{A}_a-\mathbf{A}_b\right)
        \cdot w^{'}_{ab} \mathbf{e}_{ab}
    \right\} \approx
    \left\{ \nabla \cdot \mathbf{A}_a \right\}
\]

Depending on their symmetry properties, some are more appropriate for others.
Do note that $D^k$ and $-\widetilde{\mathbf{G}}^k$ are adjoint.
(Likewise, $\widetilde{D}^k$ is adjoint with $-\mathbf{G}^k$)
So the tilde operator should be used along with the non-tilde variant.

\subsection{Renormalization}

Since particles are discreet, the kernel sum is not unity.
(Expeccially not around boundaries)

\[
    \gamma_a := J_a \left\{ 1 \right\} = \sum\limits_{b} V_b w_{ab}
\]

\[
    \gamma_a^{'} := \frac{\partial\gamma_a}{\partial\mathbf{r}_a} = 
    \sum\limits_{b} V_b w^{'}_{ab} \mathbf{e_{ab}} =
    \mathbf{G}_a\left\{1\right\}
\]

We can correct for this, and define a new \textit{renormalized} operator.

\[
    J^{\gamma}_a \left\{A_i\right\} := 
    \frac{1}{\gamma_a} \sum\limits_{b} V_b w_{ab}
\]

Which is nice since $J^{\gamma}_a \left\{1\right\}=1$.
There are also corresponding corrected operators.

\[
    \mathbf{G}^\gamma_a \left\{ A_i \right\} := \frac{1}{\gamma_a} \left(
        \mathbf{G}_a \left\{ A_i \right\} -
        J^{\gamma}_a \left\{A_i\right\} \mathbf{G}_a\left\{1\right\} \right)
\]

\[
    D^\gamma_a \left\{ \mathbf{A}_i \right\} := \frac{1}{\gamma_a} \left(
        D_a \left\{ A_i \right\} -
    J^{\gamma}_a \left\{A_i\right\} D_a\left\{\mathbf{1}\right\} \right)
\]

The k-operators lose symmetry, hence they cannot preserve flux.
They are dealt with differently.

\section{Weakly Compressible Scheme}

Violeau's Lid-Driven Cavity Flow details the use of weakly compressible scheme.
For density evolution the operator $\widetilde{D}^k$ is used to ensure no flux
between comoving points.

\[
    \frac{d\rho_a}{dt} = \frac{1}{\rho_a^{2k-1}} \sum\limits_{b}
    V_b {\left(\rho_a\rho_b\right)} \mathbf{u}_{ab} \cdot w^{'}_{ab}
    \mathbf{e}_{ab}
\]

Since $\widetilde{D}^k$ is used in the density equation,
for interparticle force $\mathbf{G}^k$ will be used

\[
    \mathbf{F}^{int}_{b \rightarrow a} = -\mathbf{F}^{int}_{b \rightarrow a} =
    -m_am_b\frac{\rho_b^{2k}p_a+\rho_a^{2k}p_b}{{\left(\rho_a\rho_b\right)}^{k+1}}
    w^{'}_{ab}\mathbf{e}_{ab}
\]

Pressure is obtained from the Taut equation

\[
    p_a = \frac{\rho_0c^2_0}{\gamma} \left[
        {\left( \frac{\rho_a}{\rho_0} \right)}^\gamma + C
    \right]
\]

The dissipative forces are

\[
    \mathbf{F}^{diss}_{b \rightarrow a} = -\mathbf{F}^{diss}_{a \rightarrow b} =
    2 \left( d + 2 \right) \bar{\mu}_{ab} V_a V_b
    \frac{\mathbf{u}_{ab} \cdot \mathbf{e}_{ab} }{r_{ab}}
    w^{'}_{ab}\mathbf{e}_{ab}
\]

Obviously, the update equations become clear

\[
    m_a \frac{d\mathbf{u}_a}{dt} = \sum\limits_b \left(
        \mathbf{F}^{int}_{b \rightarrow a} + \mathbf{F}^{diss}_{b \rightarrow a}
    \right) + \mathbf{F}^{ext}_{a}
\]

\end{document}
